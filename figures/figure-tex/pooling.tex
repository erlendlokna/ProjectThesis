\begin{center}
    \begin{figure}[H]
    \begin{tikzpicture}[
        2d-arr/.style={
            matrix of nodes,
            nodes in empty cells,
            row sep=-\pgflinewidth,
            column sep=-\pgflinewidth,
            nodes={minimum size=1cm, draw, anchor=center, scale=0.7}
        },
        pool-box/.style={
            draw=red, thick, rounded corners
        },
        scale = 0.7
    ]

    \matrix (input) [2d-arr] {
         2 & 1 & |[fill=orange!30]| 3 & 0 \\
        |[fill=orange!30]| 4 &  3 & 2 &  1 \\
        0 & 2 & 1 & |[fill=orange!30]| 4 \\
        |[fill=orange!30]| 3 & 2 &  1 & 1 \\
    };

    \node[left=of input-2-1.west, anchor=center, rotate=90] {Input Feature Map};
    
    \matrix (output) [2d-arr, right=2cm of input] {
        |[fill=blue!30]| 4 & |[fill=blue!30]| 3 \\
        |[fill=blue!30]| 3 & |[fill=blue!30]| 4 \\
    };

    \node[right=of output-2-2.east, anchor=center, rotate=90] {Max Pooled Feature Map};

    \draw[pool-box] (input-1-1.north west) rectangle (input-2-2.south east);
    \draw[pool-box] (input-1-3.north west) rectangle (input-2-4.south east);
    \draw[pool-box] (input-3-1.north west) rectangle (input-4-2.south east);
    \draw[pool-box] (input-3-3.north west) rectangle (input-4-4.south east);

    \draw[->, thick] (input) -- (output) node[midway, above] {Max Pooling};

    \end{tikzpicture}
\caption{Illustration of a max pooling operation. The input feature map is reduced in size by applying a max pooling filter with size 2x2 (red boxes), which selects the maximum value in each filter region to produce the max pooled feature map.}
\end{figure}
\end{center}